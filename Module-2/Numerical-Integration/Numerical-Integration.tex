\newpage

\chapter{Numerical Integration}
\section{Numerical Integration}

\begin{itemize}
    \item We first need to find $h$ i.e. common difference between the intervals.

    \item To determine which rule is to be used, we need to find the number of intervals.

    \item If the numeber is divisible by $1$ then it is eligible for Trapezoidal Rule.

    \item If the number is divisible by $2$ then it is eligible for Simpson's $\sfrac{1}{3^{th}}$ Rule.

    \item If the number is divisible by $3$ then it is eligible for Simpson's $\sfrac{3}{8^{th}}$ Rule.

\end{itemize}

    \[h = \frac{(Upper\ Limit) - (Lower\ Limit)}{Total\ Number\ of\ Intervals}\]
    \mybox{
        Example : Find $h$ for the following integral.
        \[ \int_{1}^{2} \frac{1}{x}, dx \]

            \[h = \frac{(Upper\ Limit) - (Lower\ Limit)}{Total\ Number\ of\ Intervals}\]
            \[h = \frac{2 - 1}{6}\]
            \[h = \frac{1}{6}\]
    }


\subsection{Trapezoidal Rule}

\subsubsection{Formula}
\begin{itemize}
    \item This formula is assuming that the function is divided into $6$ intervals.
\end{itemize}
\[ f(x) = \frac{h}{2}[(y_0 + y_6) + 2( y_1 + y_2 + y_3 + y_4 + y_5 )] \]

\subsection{Simpson's Rules}


\subsubsection{Simposn's $\sfrac{1}{3^{rd}}$ Rule}


\paragraph{Formula}
\begin{itemize}
    \item This formula is assuming that the function is divided into $6$ intervals.
\end{itemize}
\[ f(x) = \frac{h}{3}[(y_0 + y_6) + 4( y_1 + y_3 + y_5 ) + 2(y_2 + y_4)] \]

\subsubsection{Simposn's $\sfrac{3}{8^{th}}$ Rule}

\paragraph{Formula}
\begin{itemize}
    \item This formula is assuming that the function is divided into 6 intervals.
\end{itemize}
\[ f(x) = \frac{3h}{8}[(y_0 + y_6) + 4( y_1 + y_2 + y_4 + y_5 ) + 2(y_3)] \]
