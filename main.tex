\documentclass{article}

\setcounter{tocdepth}{8}

\setcounter{secnumdepth}{8}

\usepackage{amsmath}

\usepackage{blindtext}

\usepackage{titlesec}

\usepackage[a4paper, total={6in, 8in}]{geometry}

\usepackage{amssymb}

\usepackage{hyperref}

\hypersetup{
    colorlinks=true,
    linktoc=all,
    linkcolor=black,
}

\usepackage[most]{tcolorbox}

\renewcommand{\familydefault}{\sfdefault}

\clearpage

\titleformat{\chapter}[display]
  {\normalfont\bfseries}{}{0pt}{\Huge}



\makeatletter
\NewDocumentCommand{\mynote}{+O{}+m}{%
  \begingroup
  \tcbset{%
    noteshift/.store in=\mynote@shift,
    noteshift=1.5cm
  }
  \begin{tcolorbox}[
    nobeforeafter,
    enhanced,
    sharp corners,
    toprule=1pt,
    bottomrule=1pt,
    leftrule=0pt,
    rightrule=0pt,
    colback=yellow!20,
    #1,
    left skip=\mynote@shift,
    right skip=\mynote@shift,
    overlay={\node[right] (mynotenode) at ([xshift=-\mynote@shift]frame.west) {\textbf{NOTE:}};},
]
    #2
  \end{tcolorbox}
  \endgroup
  }

  \NewDocumentCommand{\mybox}{+O{}+m}{%
  \begingroup
  \tcbset{%
    noteshift/.store in=\mynote@shift,
    noteshift=1.5cm
  }
  \begin{tcolorbox}[
    nobeforeafter,
    enhanced,
    sharp corners,
    toprule=1pt,
    bottomrule=1pt,
    leftrule=1pt,
    rightrule=1pt,
    colback=gray!40,
    #1,
    left skip=\mynote@shift,
    right skip=\mynote@shift,
    % overlay={\node[right] (mynotenode) at ([xshift=-\mynote@shift]frame.west) {\textbf{NOTE:}};},
]
    #2
  \end{tcolorbox}
  \endgroup
  }
\makeatother

% \title{Iterative Techniques and Interpolation}
% \date{}
% \author{Neelanjan Mukherji}


\begin{document}

% \maketitle

\tableofcontents




\newpage
\chapter{Bisection Method}

\section{Bisection Method}

\begin{enumerate}

    \item Identify the interval $[a,b]$ that contains the root of the function $f(x)$. This means finding two points a and b such that $f(a)$ and $f(b)$ have opposite signs (i.e., $f(a) * f(b) < 0$). This interval can be obtained either graphically or algebraically.

    \item Divide the interval $[a,b]$ into two equal sub-intervals by finding the midpoint
    \begin{equation*}
        c = \frac{a + b}{2}
    \end{equation*}

    \item Evaluate the function $f(c)$ at the midpoint c. If $f(c) = 0$, then c is the root of the function and we are done.

    \item If $f(c)$ has the same sign as $f(a)$, then the root must lie in the interval $[c,b]$. So, set $a = c$ and go to step 2.

    \item If $f(c)$ has the same sign as $f(b)$, then the root must lie in the interval $[a,c]$. So, set $b = c$ and go to step 2.

    \item Repeat steps $2-5$ until you obtain an interval $[a,b]$ that is small enough or until $f(c)$ is sufficiently close to zero.

    \item The final value of c obtained is the approximate root of the function $f(x)$ within the interval $[a,b]$.

\end{enumerate}

\mynote{The Bolzano method guarantees convergence to a root of the function as long as the function is continuous on the interval $[a,b]$. However, it does not guarantee uniqueness of the root, nor does it give an estimate of the error in the approximation.}

\newpage
\chapter{Regula Falsi Method}



\section{Reguli Falsi Method}

\begin{enumerate}
  \item Identify the interval $[a,b]$ that contains the root of the function $f(x)$. This means finding two points a and b such that $f(a)$ and $f(b)$ have opposite signs (i.e., $f(a) * f(b) < 0$). This interval can be obtained either graphically or algebraically.

  \item Evaluate the function $f(a)$ and $f(b)$ at the endpoints a and b.
  
  \item Calculate the x-intercept of the straight line that connects the points $(a, f(a))$ and $(b, f(b))$. This x-intercept is given by the formula:
  \begin{equation*}
    c = \frac{a(f(b) - b(f(a)))}{f(b) - f(a)}
  \end{equation*}
  
  \item Evaluate the function $f(c)$ at the point c.
  
  \item If $f(c) = 0$, then $c$ is the root of the function and we are done.
  
  \item If $f(c)$ has the same sign as $f(a)$, then the root must lie in the interval $[c,b]$. So, set $a = c$ and go to step 2.
  
  \item If $f(c)$ has the same sign as $f(b)$, then the root must lie in the interval $[a,c]$. So, set $b = c$ and go to step 2.
  
  \item Repeat steps $2-7$ until you obtain an interval $[a,b]$ that is small enough or until $f(c)$ is sufficiently close to zero.
  
  \item The final value of c obtained is the approximate root of the function $f(x)$ within the interval $[a,b]$.
  
\mynote{The Regula Falsi method is a modified version of the Bolzano method that uses a linear approximation of the function to find the root. It also guarantees convergence to a root of the function as long as the function is continuous on the interval $[a,b]$. However, it may converge more slowly than the Bolzano method, especially for functions with steep slopes.}


\end{enumerate}

\newpage
\chapter{Newton-Raphson Method}


\section{Newton-Raphson Method}

\begin{enumerate}
\item Choose an initial guess $x_0$ for the root of the function $f(x)$.

\item Calculate the derivative $f'(x)$ of the function $f(x)$.

\item Evaluate the function $f(x)$ and its derivative $f'(x)$ at the initial guess $x_0$.

\item Calculate the next approximation $x_1$ of the root using the formula:

\begin{equation}
  x_1 = x_0 - \frac{f(x_0)}{f'(x_0)}
\end{equation}


\item Evaluate the function $f(x)$ and its derivative $f'(x)$ at the new approximation $x_1$.

\item Calculate the next approximation $x_2$ of the root using the formula:

\begin{equation}
    x_2 = x_1 - \frac{f(x_1)}{f'(x_1)}
\end{equation}

\item Repeat steps $5-6$ until you obtain an approximation $x_i$ that is sufficiently close to the root or until the maximum number of iterations is reached.

\item The final value of $x_i$ obtained is the approximate root of the function $f(x)$.

\mynote{The Newton-Raphson method can converge faster than the Bolzano and Regula Falsi methods for functions with well-behaved derivatives. However, it requires an initial guess that is sufficiently close to the root and may fail to converge or converge to a different root if the function has multiple roots or if the derivative changes sign near the root.}

\end{enumerate}

\newpage
\chapter{Jacobi Iteration Method}



\section{Jacobi Iteration Method}

\begin{enumerate}

  \item This method is applicable to the system of equation in which leading diagonal elements of co-effcient matrix are dominant(large in magnitude) in their respective rows.

  \item The system of equations is written in the form :
  \begin{equation*}
    a_{11}x + a_{12}y + a_{13}z = b_1
  \end{equation*}
  \begin{equation*}
    a_{21}x + a_{22}y + a_{23}z = b_2
  \end{equation*}
  \begin{equation*}
    a_{31}x + a_{32}y + a_{33}z = b_3
  \end{equation*}

  \mynote{
    Diagonal dominance property must be satisfied :
  
    \begin{equation*}
      |a_{11}| > |a_{12}| + |a_{13}|
    \end{equation*}
    \begin{equation*}
      |a_{22}| > |a_{21}| + |a_{23}|
    \end{equation*}
    \begin{equation*}
      |a_{33}| > |a_{31}| + |a_{32}|
    \end{equation*}  
  }

  \item Rewriting the equations for $x, y, z$ respectively :
  \begin{equation*}
    x = \frac{1}{a_{11}}(b_1 - a_{12}y - a_{13}z)
  \end{equation*}
  \begin{equation*}
    y = \frac{1}{a_{22}}(b_2 - a_{21}x - a_{23}z)
  \end{equation*}
  \begin{equation*}
    z = \frac{1}{a_{33}}(b_3 - a_{31}x - a_{32}y)
  \end{equation*}

  \item To solve the system of equations, we start with initial guesses for $x, y, z$.

  \begin {equation*}
    x_0 = 0, y_0 = 0, z_0 = 0
  \end{equation*}

  \item  Then we use the above equations to calculate the values of $x, y, z$.

  \begin{equation*}
    x = \frac{1}{a_{11}}(b_1 - a_{12}y_0 - a_{13}z_0)
  \end{equation*}
  \begin{equation*}
    y = \frac{1}{a_{22}}(b_2 - a_{21}x_0 - a_{23}z_0)
  \end{equation*}
  \begin{equation*}
    z = \frac{1}{a_{33}}(b_3 - a_{31}x_0 - a_{32}y_0)
  \end{equation*}


  \item  These values are then used to calculate the new values of $x, y, z$.
  \item  This process is repeated until the values of $x, y, z$ converge to the desired accuracy.
\end{enumerate}

\newpage
\chapter{Gauss-Seidel Iteration Method}

\section{Gauss-Seidel Iteration Method}

\begin{enumerate}

    \item This method is applicable to the system of equation in which leading diagonal elements of co-effcient matrix are dominant(large in magnitude) in their respective rows.
  
    \item The system of equations is written in the form :
    \begin{equation*}
      a_{11}x + a_{12}y + a_{13}z = b_1
    \end{equation*}
    \begin{equation*}
      a_{21}x + a_{22}y + a_{23}z = b_2
    \end{equation*}
    \begin{equation*}
      a_{31}x + a_{32}y + a_{33}z = b_3
    \end{equation*}
  
    \mynote{
      Diagonal dominance property must be satisfied :
    
      \begin{equation*}
        |a_{11}| > |a_{12}| + |a_{13}|
      \end{equation*}
      \begin{equation*}
        |a_{22}| > |a_{21}| + |a_{23}|
      \end{equation*}
      \begin{equation*}
        |a_{33}| > |a_{31}| + |a_{32}|
      \end{equation*}  
    }
  
    \item Rewriting the equations for $x, y, z$ respectively :
    \begin{equation*}
      x = \frac{1}{a_{11}}(b_1 - a_{12}y - a_{13}z)
    \end{equation*}
    \begin{equation*}
      y = \frac{1}{a_{22}}(b_2 - a_{21}x - a_{23}z)
    \end{equation*}
    \begin{equation*}
      z = \frac{1}{a_{33}}(b_3 - a_{31}x - a_{32}y)
    \end{equation*}
  
    \item To solve the system of equations, we start with initial guesses for $x, y, z$.
  
    \begin {equation*}
      x_0 = 0, y_0 = 0, z_0 = 0
    \end{equation*}
  
    \item  Then we use the above equations to calculate the values of $x, y, z$.
  
    \begin{equation*}
      x = \frac{1}{a_{11}}(b_1 - a_{12}y_0 - a_{13}z_0)
    \end{equation*}
    \begin{equation*}
      y = \frac{1}{a_{22}}(b_2 - a_{21}x_0 - a_{23}z_0)
    \end{equation*}
    \begin{equation*}
      z = \frac{1}{a_{33}}(b_3 - a_{31}x_0 - a_{32}y_0)
    \end{equation*}

\mynote[noteshift=1.5cm,colback=orange!40]{To calculate the values we use the updated values of $x, y, z$ as soon as they are calculated.}

\end{enumerate}
  
\newpage

\chapter{Finite Differences}

\section{Finite Differences}

\begin{enumerate}
  \item Finite difference is a method of approximating the derivative of a function at a point by using the function values at nearby points.
  \item The finite difference method is used to solve ordinary differential equations that have conditions imposed on the boundary rather than at the initial point.
  \item The finite difference method is used to solve partial differential equations.

    \mynote{
        \begin{enumerate}
        \item The finite difference method is used to solve ordinary differential equations that have conditions imposed on the boundary rather than at the initial point.
        \item The finite difference method is used to solve partial differential equations.
        \end{enumerate}
    }

    \begin{equation*}
        y = f(x)\\
    \end{equation*}
    Consider,
    \[ x : (a), (a+h), (a+2h), (a+3h), .... (a+nh) \]
    \[y : y_0, y_1, y_2, y_3, .... y_n\]

        \[y_1 = f(a+h)\]
        \[y_2 = f(a+2h)\]
        \[y_3 = f(a+3h)\]
        \mybox[noteshift=0.5cm,colback=lime!40]{\[y_n = f(a+nh)\]}

        Where,\\
        \[x \rightarrow Arguments\]
        \[y \rightarrow Entries\]
        \[h \rightarrow Difference\ Interval\]
\end{enumerate}



\subsection{Forward Difference($\Delta$)}

\begin{enumerate}
  \item The forward difference is defined as the difference between the function values at two consecutive points.
  \item The forward difference is denoted by $\Delta$.


        \[\Delta y = y_1 - y_0\]
        \[\Delta y_0 = f(a+h) - f(a)\]
        \[\Delta y_1 = f(a+2h) - f(a+h)\]
        \[\Delta y_2 = f(a+3h) - f(a+2h)\]
        \[\Delta y_3 = f(a+4h) - f(a+3h)\]
        \[\Delta y_n = f(a+(n+1)h) - f(a+nh)\]
\end{enumerate}

\subsubsection{$n^{th}$ Forward Difference($\Delta^n$)}

\[\Delta^n(\Delta y_0) = \Delta^n(y_1 - y_0)\]
$\therefore$ \[\Delta^n y_0 = \Delta^n y_1 - \Delta^n y_0\]

\mybox{
    Example : 
    \[\Delta(\Delta y_0) = \Delta(y_1 - y_0)\]
    \[\Delta^2 y_0 = \Delta y_1 - \Delta y_0\]
}

\subsubsection{Forward Difference Table}

\begin{enumerate}
  \item The forward difference table is a table that is used to calculate the forward difference of a function.
\end{enumerate}


% \begin{table}
% \centering
\begin{center}
\begin{tabular}{|c|c|c|c|c|c|c|}
    \hline
    $x$ & $y$ & $\Delta$ & $\Delta^2$ & $\Delta^3$ & $\Delta^4$ & $\Delta^5$\\
    \hline
    $a$ & $y_0$ & & & & &\\
    \hline
    $a + h$ & $y_1$ & $\Delta y_0$ & & & &\\
    \hline
    $a + 2h$ & $y_2$ & $\Delta y_1$ & $\Delta^2 y_0$ & & &\\
    \hline
    $a + 3h$ & $y_3$ & $\Delta y_2$ & $\Delta^2 y_1$ & $\Delta^3 y_0$ & &\\
    \hline
    $a + 4h$ & $y_4$ & $\Delta y_3$ & $\Delta^2 y_2$ & $\Delta^3 y_1$ & $\Delta^4 y_0$ &\\
    \hline
    $a + 5h$ & $y_5$ & $\Delta y_4$ & $\Delta^2 y_3$ & $\Delta^3 y_2$ & $\Delta^4 y_1$ & $\Delta^5 y_0$\\
    \hline
\end{tabular}
\end{center}
% \end{table}


\subsection{Backward Difference($\nabla$)}

\begin{enumerate}
  \item The backward difference is defined as the difference between the function values at two consecutive points.
  \item The backward difference is denoted by $\nabla$.

  \[\nabla y = y_0 - {y}_{-1}\]
  \[\nabla y_0 = f(a-h) - f(a)\]
  \[\nabla y_1 = f(a-2h) - f(a-h)\]
  \[\nabla y_2 = f(a-3h) - f(a-2h)\]
  \[\nabla y_3 = f(-4h) - f(a-3h)\]
  \[\nabla y_n = f(a-(n-1)h) - f(a-nh)\]

\end{enumerate}

\subsubsection{$n^{th}$ Backward Difference($\nabla^n$)}

\[\nabla^n(\nabla y_0) = \nabla^n(y_0 - y_{-1})\]
$\therefore$ \[\nabla^n y_0 = \nabla^n y_0 - \nabla^n y_{-1}\]

\mybox{
    Example : 
    \[\nabla(\nabla y_0) = \nabla(y_0 - y_{-1})\]
    \[\nabla^2 y_0 = \nabla y_0 - \nabla y_{-1}\]
}

\subsubsection{Backward Difference Table}

\begin{enumerate}
  \item The backward difference table is a table that is used to calculate the backward difference of a function.
\end{enumerate}

% \begin{table}
% \centering
\begin{center}
\begin{tabular}{|c|c|c|c|c|c|c|}
    \hline
    $x$ & $y$ & $\nabla$ & $\nabla^2$ & $\nabla^3$ & $\nabla^4$ & $\nabla^5$\\
    \hline
    $a - 5h$ & ${y}_{-5}$ & & & & &\\
    \hline
    $a - 4h$ & ${y}_{-4}$ & $\nabla {y}_{-4}$ & & & &\\
    \hline
    $a - 3h$ & ${y}_{-3}$ & $\nabla {y}_{-3}$ & $\nabla^2 {y}_{-3}$ & & &\\
    \hline
    $a - 2h$ & ${y}_{-2}$ & $\nabla {y}_{-2}$ & $\nabla^2 y_{-2}$ & $\nabla^3 y_{-2}$ & &\\
    \hline
    $a - 1h$ & ${y}_{-1}$ & $\nabla {y}_{-1}$ & $\nabla^2 y_{-1}$ & $\nabla^3 y_{-1}$ & $\nabla^4 {y}_{-1}$ &\\
    \hline
    $a$ & $y_0$ & $\nabla y_0$ & $\nabla^2 y_0$ & $\nabla^3 y_0$ & $\nabla^4 y_0$ & $\nabla^5 y_0$\\
    \hline
\end{tabular}
\end{center}
% \end{table}
\subsection{Shift Operator($E$)}

\begin{enumerate}
  \item The shift operator is defined as the difference between the function values at two consecutive points.
  \item The shift operator is denoted by $E$.

  \mybox{
      General Relation : \\
    
      \[E^m y_n = {y}_{m + n}\]

  }


  \[E f(a) = f(a +h) \]
  \[E f(a +h) = f(a +2h) \]
  \[E f(a +2h) = f(a +3h) \]
  \[E f(a +3h) = f(a +4h) \]
  \[E f(a +4h) = f(a +5h) \]

  \mybox{
    Example :
    \[E sinx = sin(x + h) \]
    \[E e^{2x} = e^{2(x+h)}\]
  }

\end{enumerate}

\subsubsection{$n^{th}$ Shift Operator($E^n$)}

\[E^n f(a) = f(a + nh)\]

\mybox{
    Example :
    \[E^2 f(a) = f(a + 2h)\]
    \[E^3 f(a) = f(a + 3h)\]
    \[E^2 sinx = sin(x + 2h) \]
    \[E^3 e^{2x} = e^{2(x+3h)}\]
}


\subsubsection{Some Examples of Shift Operator($E^{1}$)}

\[E^{1} y_0 = y_1\]
\[E^{1} y_1 = y_2\]
\[E^{1} y_5 = y_6\]
\[E^{2} y_0 = y_2\]
\[E^{2} y_3 = y_5\]
\[E^{3} y_6 = y_9\]
\[E^{4} y_4 = y_8\]
\subsection{Inverse Operator($E^{-1}$)}

\begin{enumerate}
  \item The inverse operator is defined as the difference between the function values at two consecutive points.
  \item The inverse operator is denoted by $E^{-1}$.

  \mybox{
    General Relation : \\
  
    \[{E}^{-m} y_n = {y}_{n - m}\]

}

  \[E^{-1} f(a - h) = f(a) \]
  \[E^{-1} f(a + 2h) = f(a + h) \]
  \[E^{-1} f(a + 3h) = f(a + 2h) \]
  \[E^{-1} f(a + 4h) = f(a + 3h) \]
  \[E^{-1} f(a + 5h) = f(a + 4h) \]

  \mybox{
    Example :
    \[E^{-1} sin(x) = sin(x-h) \]
    \[E^{-1} e^{2(x+h)} = e^{2(x-h)}\]
  }

  
\end{enumerate}

\subsubsection{$n^{th}$ Inverse Operator($E^{-n}$)}

\[E^{-n} f(a + nh) = f(a)\]

\mybox{
    Example :
    \[E^{-2} f(a + h) = f(a - h)\]
    \[E^{-3} f(a + 3h) = f(a)\]
    \[E^{-4} f(a) = f(a -4h)\]
}

\subsubsection{Some Examples of Inverse Operator($E^{-1}$)}

\[E^{-1} y_0 = {y}_{-1}\]
\[E^{-1} y_1 = y_0\]
\[E^{-1} y_5 = y_4\]
\[E^{-2} y_0 = {y}_{-2}\]
\[E^{-2} y_3 = y_1\]
\[E^{-3} y_6 = {y}_{-3}\]
\newpage
\chapter{Polynomial Interpolation}

\section{Polynomial Interpolation}

The technique or method of estimating unknown values from given set of observation is known as Polynomial Interpolation.

There are two types of Polynomial Interpolation:

\begin{enumerate}
    \item Equal interval
    \item Unequal interval
\end{enumerate}

\subsection{Equal Interval}

Here we use the following formulae for estimating interpolation with equal interval :
\begin{enumerate}
    \item \textbf{Newton's Forward Formula}
    \item \textbf{Newton's Backward Formula}
    \item \textbf{Gauss's Forward Formula}
    \item \textbf{Gauss's Backward Formula}
    \item \textbf{Stirling's Formula}
\end{enumerate}

\subsection{Unequal Interval}

Here we use the following formulae for estimating interpolation with unequal interval :
\begin{enumerate}
    \item \textbf{Lagrange's Formula}
    \item \textbf{Newton's Divided Difference Formula}
\end{enumerate}

\subsection{Newton's Forward Formula}

\begin{enumerate}
    \item This formula is used for estimating the values from the top to bottom in a difference table.
    \item As the top values must be near to the desired interval.
\end{enumerate}

To solve and find the accurate value we must find out the $\Delta f(x)$ and $\Delta^2 f(x)$ and so on.

This is possible by creating a difference table.

Example:

% \begin{table}
% \centering
\begin{center}
    \begin{tabular}{|c|c|c|c|c|c|c|}
        \hline
        $x$ & $y$ & $\Delta$ & $\Delta^2$ & $\Delta^3$ & $\Delta^4$ & $\Delta^5$\\
        \hline
        $a$ & $y_0$ & & & & &\\
        \hline
        $a + h$ & $y_1$ & $\Delta y_0$ & & & &\\
        \hline
        $a + 2h$ & $y_2$ & $\Delta y_1$ & $\Delta^2 y_0$ & & &\\
        \hline
        $a + 3h$ & $y_3$ & $\Delta y_2$ & $\Delta^2 y_1$ & $\Delta^3 y_0$ & &\\
        \hline
        $a + 4h$ & $y_4$ & $\Delta y_3$ & $\Delta^2 y_2$ & $\Delta^3 y_1$ & $\Delta^4 y_0$ &\\
        \hline
        $a + 5h$ & $y_5$ & $\Delta y_4$ & $\Delta^2 y_3$ & $\Delta^3 y_2$ & $\Delta^4 y_1$ & $\Delta^5 y_0$\\
        \hline
    \end{tabular}
    \end{center}
% \end{table}

Where,\\
\[x \rightarrow Arguments\]
\[y \rightarrow Entries\]
\[h \rightarrow Difference\ Interval\]
\[u \rightarrow Interpolation\ Difference\]


\subsubsection{Formula}

\[f(a\ + (u)h) = f(a) + \frac{u}{1!}(\Delta f(a)) + \frac{u(u-1)}{2!}(\Delta^2 f(a)) + \frac{u(u-1)(u-2)}{3!}(\Delta^3 f(a)) + \frac{u(u-1)(u-2)(u-3)}{4!}(\Delta^4 f(a)) \cdots\]
\subsection{Newton's Backward Formula}

\begin{enumerate}
    \item This formula is used for estimating the values from the bottom to top in a difference table.
    \item As the bottom values must be near to the desired interval.
\end{enumerate}

To solve and find the accurate value we must find out the $\Delta f(x)$ and $\Delta^2 f(x)$ and so on.

This is possible by creating a difference table.

Example:

% \begin{table}
% \centering
\begin{center}
    \begin{tabular}{|c|c|c|c|c|c|c|}
        \hline
        $x$ & $y$ & $\nabla$ & $\nabla^2$ & $\nabla^3$ & $\nabla^4$ & $\nabla^5$\\
        \hline
        $a - 5h$ & ${y}_{-5}$ & & & & &\\
        \hline
        $a - 4h$ & ${y}_{-4}$ & $\nabla {y}_{-4}$ & & & &\\
        \hline
        $a - 3h$ & ${y}_{-3}$ & $\nabla {y}_{-3}$ & $\nabla^2 {y}_{-3}$ & & &\\
        \hline
        $a - 2h$ & ${y}_{-2}$ & $\nabla {y}_{-2}$ & $\nabla^2 y_{-2}$ & $\nabla^3 y_{-2}$ & &\\
        \hline
        $a - 1h$ & ${y}_{-1}$ & $\nabla {y}_{-1}$ & $\nabla^2 y_{-1}$ & $\nabla^3 y_{-1}$ & $\nabla^4 {y}_{-1}$ &\\
        \hline
        $a$ & $y_0$ & $\nabla y_0$ & $\nabla^2 y_0$ & $\nabla^3 y_0$ & $\nabla^4 y_0$ & $\nabla^5 y_0$\\
        \hline
    \end{tabular}
    \end{center}
    % \end{table}

Where,\\
\[x \rightarrow Arguments\]
\[y \rightarrow Entries\]
\[h \rightarrow Difference\ Interval\]
\[u \rightarrow Interpolation\ Difference\]


\subsubsection{Formula}

\[f(a\ + (u)h) = f(a) + \frac{u}{1!}(\nabla f(a)) + \frac{u(u+1)}{2!}(\nabla^2 f(a)) + \frac{u(u+1)(u+2)}{3!}(\nabla^3 f(a)) + \frac{u(u+1)(u+2)(u+3)}{4!}(\nabla^4 f(a)) \ldots\]
\subsection{Gauss' Forward Formula}

\begin{enumerate}
    \item This formula is used for estimating the values from the top to bottom in a difference table.
    \item As the top values must be near to the desired interval.
    \item Used to centre interpolation difference.
\end{enumerate}

To solve and find the accurate value we must find out the $\Delta f(x)$ and $\Delta^2 f(x)$ and so on.

This is possible by creating a difference table.

Example:

% \begin{table}
% \centering
\begin{center}
    \begin{tabular}{|c|c|c|c|c|c|c|}
        \hline
        $x$ & $y$ & $\Delta$ & $\Delta^2$ & $\Delta^3$ & $\Delta^4$ & $\Delta^5$\\
        \hline
        $a$ & $y_0$ & & & & &\\
        \hline
        $a + h$ & $y_1$ & $\Delta y_0$ & & & &\\
        \hline
        $a + 2h$ & $y_2$ & $\Delta y_1$ & $\Delta^2 y_0$ & & &\\
        \hline
        $a + 3h$ & $y_3$ & $\Delta y_2$ & $\Delta^2 y_1$ & $\Delta^3 y_0$ & &\\
        \hline
        $a + 4h$ & $y_4$ & $\Delta y_3$ & $\Delta^2 y_2$ & $\Delta^3 y_1$ & $\Delta^4 y_0$ &\\
        \hline
        $a + 5h$ & $y_5$ & $\Delta y_4$ & $\Delta^2 y_3$ & $\Delta^3 y_2$ & $\Delta^4 y_1$ & $\Delta^5 y_0$\\
        \hline
    \end{tabular}
    \end{center}
% \end{table}

Where,\\
\[x \rightarrow Arguments\]
\[y \rightarrow Entries\]
\[h \rightarrow Difference\ Interval\]
\[u \rightarrow Interpolation\ Difference\]


\subsubsection{Formula}

\[f(a + uh) = y_0 + \frac{u}{1!}(\Delta {y}_{0}) + \frac{u(u-1)}{2!}(\Delta^2 {y}_{-1}) + \frac{u(u-1)(u+1)}{3!}(\Delta^3 {y}_{-1}) + \frac{u(u-1)(u+1)(u-2)}{4!}(\Delta^4 {y}_{-2}) \cdots\]
\subsection{Guass' Backward Formula}

\begin{enumerate}
    \item This formula is used for estimating the values from the bottom to top in a difference table.
    \item As the bottom values must be near to the desired interval.
\end{enumerate}

To solve and find the accurate value we must find out the $\Delta f(x)$ and $\Delta^2 f(x)$ and so on.

This is possible by creating a difference table.

Example:

% \begin{table}
% \centering
\begin{center}
    \begin{tabular}{|c|c|c|c|c|c|c|}
        \hline
        $x$ & $y$ & $\nabla$ & $\nabla^2$ & $\nabla^3$ & $\nabla^4$ & $\nabla^5$\\
        \hline
        $a - 5h$ & ${y}_{-5}$ & & & & &\\
        \hline
        $a - 4h$ & ${y}_{-4}$ & $\nabla {y}_{-4}$ & & & &\\
        \hline
        $a - 3h$ & ${y}_{-3}$ & $\nabla {y}_{-3}$ & $\nabla^2 {y}_{-3}$ & & &\\
        \hline
        $a - 2h$ & ${y}_{-2}$ & $\nabla {y}_{-2}$ & $\nabla^2 y_{-2}$ & $\nabla^3 y_{-2}$ & &\\
        \hline
        $a - 1h$ & ${y}_{-1}$ & $\nabla {y}_{-1}$ & $\nabla^2 y_{-1}$ & $\nabla^3 y_{-1}$ & $\nabla^4 {y}_{-1}$ &\\
        \hline
        $a$ & $y_0$ & $\nabla y_0$ & $\nabla^2 y_0$ & $\nabla^3 y_0$ & $\nabla^4 y_0$ & $\nabla^5 y_0$\\
        \hline
    \end{tabular}
    \end{center}
    % \end{table}

Where,\\
\[x \rightarrow Arguments\]
\[y \rightarrow Entries\]
\[h \rightarrow Difference\ Interval\]
\[u \rightarrow Interpolation\ Difference\]


\subsubsection{Formula}

\[f(a + uh) = y_0 + \frac{u}{1!}(\Delta {y}_{0}) + \frac{u(u-1)}{2!}(\Delta^2 {y}_{-1}) + \frac{u(u-1)(u+1)}{3!}(\Delta^3 {y}_{-1}) + \frac{u(u-1)(u+1)(u+2)}{4!}(\Delta^4 {y}_{-2}) \cdots\]

\newpage
\subsection{Lagrange's Interpolation}

\subsubsection{Formula}

\begin{center}
\begin{tabular}{|c|c|c|c|c|}
    \hline
    $x$ & $a_1$ & $a_2$ & $a_3$ & $a_4$ \\
    \hline
    $y$ & $b_1$ & $b_2$ & $b_3$ & $b_4$ \\
    \hline
\end{tabular}
\end{center}

\begin{align*}
    f(x) &= \frac{(x - a_2)(x - a_3)(x - a_4)}{(a_1 - a_2)(a_1 - a_3)(a_1 - a_4)}(b_1) + \frac{(x - a_1)(x - a_3)(x - a_4)}{(a_2 - a_1)(a_2 - a_3)(a_2 - a_4)}(b_2) \\ \notag
    &\quad + \frac{(x - a_1)(x - a_2)(x - a_4)}{(a_3 - a_1)(a_3 - a_2)(a_3 - a_4)}(b_3) + \frac{(x - a_1)(x - a_2)(x - a_3)}{(a_4 - a_1)(a_4 - a_2)(a_4 - a_3)}(b_4) \notag
\end{align*}

\mybox{Example : Find the value of $f(x)$ at $x = 10$ from the following table using Lagrange's Interpolation Formula.

\begin{center}
\begin{tabular}{|c|c|c|c|c|}
    \hline
    $x$ & 5 & 6 & 9 & 11 \\
    \hline
    $y$ & 12 & 13 & 14 & 16 \\
    \hline
\end{tabular}
\end{center}

Sol. :

\begin{align*}
    f(x) &= \frac{(x - 6)(x - 9)(x - 11)}{(5 - 6)(5 - 9)(5 - 11)}(12) + \frac{(x - 5)(x - 9)(x - 11)}{(6 - 5)(5 - 9)(6 - 11)}(13) \\ \notag
    &\quad + \frac{(x - 5)(x - 6)(x - 11)}{(9 - 6)(9 - 9)(9 - 11)}(14) + \frac{(x - 5)(x - 9)(x - 11)}{(11 - 5)(11 - 6)(11 - 9)}(16) \notag
\end{align*}

\begin{align*}
    f(10) &= \frac{(10 - 6)(10 - 9)(10 - 11)}{(5 - 6)(5 - 9)(5 - 11)}(12) + \frac{(10 - 5)(10 - 9)(10 - 11)}{(6 - 5)(5 - 9)(6 - 11)}(13) \\ \notag
    &\quad + \frac{(10 - 5)(10 - 6)(10 - 11)}{(9 - 6)(9 - 9)(9 - 11)}(14) + \frac{(10 - 5)(10 - 9)(10 - 11)}{(11 - 5)(11 - 6)(11 - 9)}(16) \notag
\end{align*}

\begin{align*}
    f(10) &= \frac{(10 - 6)(10 - 9)(10 - 11)}{-24}(12) + \frac{(10 - 5)(10 - 9)(10 - 11)}{15}(13) \\ \notag
    &\quad + \frac{(10 - 5)(10 - 6)(10 - 11)}{-24}(14) + \frac{(10 - 5)(10 - 9)(10 - 11)}{60}(16) \notag
\end{align*}

\[f(x) = 4 + 12.666\]

\[\Longrightarrow f(x) = 16.666\]}
\newpage
\subsection{Newton's Divided Difference}

\subsubsection{Formula}
\[f(x) = f(x_0) + (x - x_0) \Delta f(x_0) + (x - x_0)(x - x_1) \Delta^2 f(x_0) + (x - x_0)(x - x_1)(x - x_2) \Delta^3 f(x_0) \cdots\]

\begin{center}
    \begin{tabular}{|c|c|c|c|c|c|c|}
        \hline
        $x$ & $f(x)$ & $\Delta f(x)$ & $\Delta^2 f(x)$ & $\Delta^3 f(x)$\\
        \hline
        $x_0$ & $a_0$ & & &\\
        \hline
        $x_1$ & $a_1$ & $\ \Delta f(x_1) = \frac{a_1 - a_0}{x_1 - x_0}$ & &\\
        \hline
        $x_2$ & $a_2$ & $ \Delta f(x_2) = \frac{a_2 - a_1}{x_2 - x_1}$ & $\Delta^2 f(x_1) = \frac{\Delta f(x_2) - \Delta f(x_1)}{x_2 - x_1}$ & \\
        \hline
        $x_3$ & $a_3$ & $\ \Delta f(x_3) = \frac{a_3 - a_2}{x_3 - x_2}$ & $\Delta^2 f(x_2) = \frac{\Delta f(x_3) - \Delta f(x_2)}{x_3 - x_2}$ & $\Delta^3 f(x_1) = \frac{\Delta^2 f(x_2) - \Delta^2 f(x_1)}{x_3 - x_2}$\\
        \hline
    \end{tabular}
\end{center}

\mybox{Example : Find the value of $f(x)$ at $x = 10$ from the following table using Newton's Divided Difference Formula.

\begin{center}
\begin{tabular}{|c|c|c|c|c|}
    \hline
    $x$ & 5 & 6 & 9 & 11 \\
    \hline
    $y$ & 12 & 13 & 14 & 16 \\
    \hline
\end{tabular}
\end{center}

Sol. :

\begin{center}
\begin{tabular}{|c|c|c|c|c|}
    \hline
    $x$ & $y$ & $\Delta y$ & $\Delta^2 y$ & $\Delta^3 y$ \\
    \hline
    5 & 12 & & & \\
    \hline
    6 & 13 & $\Delta y = \frac{13 - 12}{6 - 5} = 1$ & & \\
    \hline
    9 & 14 & $\Delta y = \frac{14 - 13}{9 - 6} = \frac{1}{3}$ & $\Delta^2 y = \frac{\frac{1}{3} - 1}{9 - 5} = -\frac{1}{6}$ & \\
    \hline
    11 & 16 & $\Delta y = \frac{16 - 14}{11 - 9} = 1$ & $\Delta^2 y = \frac{1 - \frac{1}{3}}{11 - 6} = \frac{1}{30}$ & $\Delta^3 y = \frac{\frac{1}{30} - \left(-\frac{1}{6}\right)}{11 - 5} = \frac{1}{60}$ \\
    \hline
\end{tabular}
\end{center}

\[f(x) = 12 + (x - 5)(1) + (x - 5)(x - 6)(-0.222) + (x - 5)(x - 6)(x - 9)(0.09258)\]
\[\Longrightarrow f(x) = 16.666\]
}





\end{document}