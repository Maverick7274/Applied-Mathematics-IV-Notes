\subsection{Shift Operator($E$)}

\begin{enumerate}
  \item The shift operator is defined as the difference between the function values at two consecutive points.
  \item The shift operator is denoted by $E$.

  \mybox{
      General Relation : \\
    
      \[E^m y_n = {y}_{m + n}\]

  }


  \[E f(a) = f(a +h) \]
  \[E f(a +h) = f(a +2h) \]
  \[E f(a +2h) = f(a +3h) \]
  \[E f(a +3h) = f(a +4h) \]
  \[E f(a +4h) = f(a +5h) \]

  \mybox{
    Example :
    \[E sinx = sin(x + h) \]
    \[E e^{2x} = e^{2(x+h)}\]
  }

\end{enumerate}

\subsubsection{$n^{th}$ Shift Operator($E^n$)}

\[E^n f(a) = f(a + nh)\]

\mybox{
    Example :
    \[E^2 f(a) = f(a + 2h)\]
    \[E^3 f(a) = f(a + 3h)\]
    \[E^2 sinx = sin(x + 2h) \]
    \[E^3 e^{2x} = e^{2(x+3h)}\]
}


\subsubsection{Some Examples of Shift Operator($E^{1}$)}

\[E^{1} y_0 = y_1\]
\[E^{1} y_1 = y_2\]
\[E^{1} y_5 = y_6\]
\[E^{2} y_0 = y_2\]
\[E^{2} y_3 = y_5\]
\[E^{3} y_6 = y_9\]
\[E^{4} y_4 = y_8\]