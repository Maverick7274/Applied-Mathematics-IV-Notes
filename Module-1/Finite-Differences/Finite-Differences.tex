\newpage

\chapter{Finite Differences}

\section{Finite Differences}

\begin{enumerate}
  \item Finite difference is a method of approximating the derivative of a function at a point by using the function values at nearby points.
  \item The finite difference method is used to solve ordinary differential equations that have conditions imposed on the boundary rather than at the initial point.
  \item The finite difference method is used to solve partial differential equations.

    \mynote{
        \begin{enumerate}
        \item The finite difference method is used to solve ordinary differential equations that have conditions imposed on the boundary rather than at the initial point.
        \item The finite difference method is used to solve partial differential equations.
        \end{enumerate}
    }

    \begin{equation*}
        y = f(x)\\
    \end{equation*}
    Consider,
    \[ x : (a), (a+h), (a+2h), (a+3h), .... (a+nh) \]
    \[y : y_0, y_1, y_2, y_3, .... y_n\]

        \[y_1 = f(a+h)\]
        \[y_2 = f(a+2h)\]
        \[y_3 = f(a+3h)\]
        \mybox[noteshift=0.5cm,colback=lime!40]{\[y_n = f(a+nh)\]}

        Where,\\
        \[x \rightarrow Arguments\]
        \[y \rightarrow Entries\]
        \[h \rightarrow Difference\ Interval\]
\end{enumerate}



\subsection{Forward Difference($\Delta$)}

\begin{enumerate}
  \item The forward difference is defined as the difference between the function values at two consecutive points.
  \item The forward difference is denoted by $\Delta$.


        \[\Delta y = y_1 - y_0\]
        \[\Delta y_0 = f(a+h) - f(a)\]
        \[\Delta y_1 = f(a+2h) - f(a+h)\]
        \[\Delta y_2 = f(a+3h) - f(a+2h)\]
        \[\Delta y_3 = f(a+4h) - f(a+3h)\]
        \[\Delta y_n = f(a+(n+1)h) - f(a+nh)\]
\end{enumerate}

\subsubsection{$n^{th}$ Forward Difference($\Delta^n$)}

\[\Delta^n(\Delta y_0) = \Delta^n(y_1 - y_0)\]
$\therefore$ \[\Delta^n y_0 = \Delta^n y_1 - \Delta^n y_0\]

\mybox{
    Example : 
    \[\Delta(\Delta y_0) = \Delta(y_1 - y_0)\]
    \[\Delta^2 y_0 = \Delta y_1 - \Delta y_0\]
}

\subsubsection{Forward Difference Table}

\begin{enumerate}
  \item The forward difference table is a table that is used to calculate the forward difference of a function.
\end{enumerate}


% \begin{table}
% \centering
\begin{center}
\begin{tabular}{|c|c|c|c|c|c|c|}
    \hline
    $x$ & $y$ & $\Delta$ & $\Delta^2$ & $\Delta^3$ & $\Delta^4$ & $\Delta^5$\\
    \hline
    $a$ & $y_0$ & & & & &\\
    \hline
    $a + h$ & $y_1$ & $\Delta y_0$ & & & &\\
    \hline
    $a + 2h$ & $y_2$ & $\Delta y_1$ & $\Delta^2 y_0$ & & &\\
    \hline
    $a + 3h$ & $y_3$ & $\Delta y_2$ & $\Delta^2 y_1$ & $\Delta^3 y_0$ & &\\
    \hline
    $a + 4h$ & $y_4$ & $\Delta y_3$ & $\Delta^2 y_2$ & $\Delta^3 y_1$ & $\Delta^4 y_0$ &\\
    \hline
    $a + 5h$ & $y_5$ & $\Delta y_4$ & $\Delta^2 y_3$ & $\Delta^3 y_2$ & $\Delta^4 y_1$ & $\Delta^5 y_0$\\
    \hline
\end{tabular}
\end{center}
% \end{table}


\subsection{Backward Difference($\nabla$)}

\begin{enumerate}
  \item The backward difference is defined as the difference between the function values at two consecutive points.
  \item The backward difference is denoted by $\nabla$.

  \[\nabla y = y_0 - {y}_{-1}\]
  \[\nabla y_0 = f(a-h) - f(a)\]
  \[\nabla y_1 = f(a-2h) - f(a-h)\]
  \[\nabla y_2 = f(a-3h) - f(a-2h)\]
  \[\nabla y_3 = f(-4h) - f(a-3h)\]
  \[\nabla y_n = f(a-(n-1)h) - f(a-nh)\]

\end{enumerate}

\subsubsection{$n^{th}$ Backward Difference($\nabla^n$)}

\[\nabla^n(\nabla y_0) = \nabla^n(y_0 - y_{-1})\]
$\therefore$ \[\nabla^n y_0 = \nabla^n y_0 - \nabla^n y_{-1}\]

\mybox{
    Example : 
    \[\nabla(\nabla y_0) = \nabla(y_0 - y_{-1})\]
    \[\nabla^2 y_0 = \nabla y_0 - \nabla y_{-1}\]
}

\subsubsection{Backward Difference Table}

\begin{enumerate}
  \item The backward difference table is a table that is used to calculate the backward difference of a function.
\end{enumerate}

% \begin{table}
% \centering
\begin{center}
\begin{tabular}{|c|c|c|c|c|c|c|}
    \hline
    $x$ & $y$ & $\nabla$ & $\nabla^2$ & $\nabla^3$ & $\nabla^4$ & $\nabla^5$\\
    \hline
    $a - 5h$ & ${y}_{-5}$ & & & & &\\
    \hline
    $a - 4h$ & ${y}_{-4}$ & $\nabla {y}_{-4}$ & & & &\\
    \hline
    $a - 3h$ & ${y}_{-3}$ & $\nabla {y}_{-3}$ & $\nabla^2 {y}_{-3}$ & & &\\
    \hline
    $a - 2h$ & ${y}_{-2}$ & $\nabla {y}_{-2}$ & $\nabla^2 y_{-2}$ & $\nabla^3 y_{-2}$ & &\\
    \hline
    $a - 1h$ & ${y}_{-1}$ & $\nabla {y}_{-1}$ & $\nabla^2 y_{-1}$ & $\nabla^3 y_{-1}$ & $\nabla^4 {y}_{-1}$ &\\
    \hline
    $a$ & $y_0$ & $\nabla y_0$ & $\nabla^2 y_0$ & $\nabla^3 y_0$ & $\nabla^4 y_0$ & $\nabla^5 y_0$\\
    \hline
\end{tabular}
\end{center}
% \end{table}
\subsection{Shift Operator($E$)}

\begin{enumerate}
  \item The shift operator is defined as the difference between the function values at two consecutive points.
  \item The shift operator is denoted by $E$.

  \mybox{
      General Relation : \\
    
      \[E^m y_n = {y}_{m + n}\]

  }


  \[E f(a) = f(a +h) \]
  \[E f(a +h) = f(a +2h) \]
  \[E f(a +2h) = f(a +3h) \]
  \[E f(a +3h) = f(a +4h) \]
  \[E f(a +4h) = f(a +5h) \]

  \mybox{
    Example :
    \[E sinx = sin(x + h) \]
    \[E e^{2x} = e^{2(x+h)}\]
  }

\end{enumerate}

\subsubsection{$n^{th}$ Shift Operator($E^n$)}

\[E^n f(a) = f(a + nh)\]

\mybox{
    Example :
    \[E^2 f(a) = f(a + 2h)\]
    \[E^3 f(a) = f(a + 3h)\]
    \[E^2 sinx = sin(x + 2h) \]
    \[E^3 e^{2x} = e^{2(x+3h)}\]
}


\subsubsection{Some Examples of Shift Operator($E^{1}$)}

\[E^{1} y_0 = y_1\]
\[E^{1} y_1 = y_2\]
\[E^{1} y_5 = y_6\]
\[E^{2} y_0 = y_2\]
\[E^{2} y_3 = y_5\]
\[E^{3} y_6 = y_9\]
\[E^{4} y_4 = y_8\]
\subsection{Inverse Operator($E^{-1}$)}

\begin{enumerate}
  \item The inverse operator is defined as the difference between the function values at two consecutive points.
  \item The inverse operator is denoted by $E^{-1}$.

  \mybox{
    General Relation : \\
  
    \[{E}^{-m} y_n = {y}_{n - m}\]

}

  \[E^{-1} f(a - h) = f(a) \]
  \[E^{-1} f(a + 2h) = f(a + h) \]
  \[E^{-1} f(a + 3h) = f(a + 2h) \]
  \[E^{-1} f(a + 4h) = f(a + 3h) \]
  \[E^{-1} f(a + 5h) = f(a + 4h) \]

  \mybox{
    Example :
    \[E^{-1} sin(x) = sin(x-h) \]
    \[E^{-1} e^{2(x+h)} = e^{2(x-h)}\]
  }

  
\end{enumerate}

\subsubsection{$n^{th}$ Inverse Operator($E^{-n}$)}

\[E^{-n} f(a + nh) = f(a)\]

\mybox{
    Example :
    \[E^{-2} f(a + h) = f(a - h)\]
    \[E^{-3} f(a + 3h) = f(a)\]
    \[E^{-4} f(a) = f(a -4h)\]
}

\subsubsection{Some Examples of Inverse Operator($E^{-1}$)}

\[E^{-1} y_0 = {y}_{-1}\]
\[E^{-1} y_1 = y_0\]
\[E^{-1} y_5 = y_4\]
\[E^{-2} y_0 = {y}_{-2}\]
\[E^{-2} y_3 = y_1\]
\[E^{-3} y_6 = {y}_{-3}\]